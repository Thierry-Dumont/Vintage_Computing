\documentclass[11pt]{amsart}
\usepackage[utf8]{inputenc}
%\usepackage[T1]{fontenc}
\usepackage{lmodern}
\usepackage{amsmath}
\usepackage{amsfonts}
\usepackage{amssymb,pgfpages}
\title{Bairstow's Method (Sir Leonard Bairstow,
    1880--1963)}

\begin{document}
\maketitle


  \begin{center}
    \textcolor{blue}{\large How to find roots of a polynomial  $P(x)$
      with real coefficients 
        \emph{without using complex arithmetic.}}
  \end{center}
   \subsection*{The idea:}    

  search for a quadratic divisor of $P$. If the degree of the quotient $Q$
  is greater or equal 2, reiterate with this procedure with $P=Q$.

  \subsubsection*{How to find a quadratic divisor of $p$?} 
 

  Given real numbers \textcolor{blue}{$B$} and \textcolor{blue}{$C$},
  we have:

  $$P(x)= (x^2+\textcolor{blue}{B}x+\textcolor{blue}{C})\ Q(x)+ (\textcolor{blue}{R}x+\textcolor{blue}{S}),$$

  This defines a mapping:
  
$$\mathcal{F}: \begin{pmatrix}B \\ C  \end{pmatrix}
\mapsto  \begin{pmatrix}R\\S\end{pmatrix}.$$

  Just cancel $\mathcal{F}(B,C)$ using Newton's method.

  \subsection*{The beauty of this method.}
  
How to compute the partial derivatives of  $\mathcal{F}$? The idea of
Bairstow is  nice and simple:

Once $P$ as been divided by
$(x^2+\textcolor{blue}{B}x+\textcolor{blue}{C})$, that is:
$$P(x)= (x^2+\textcolor{blue}{B}x+\textcolor{blue}{C})\ Q(x)+ (\textcolor{blue}{R}x+\textcolor{blue}{S}),$$

just cancel the first variations of $P$. This gives for $C$:

$$0 = \frac{\partial P}{\partial C}= (x^2+Bx+C)\ \frac{\partial
Q}{\partial C}+ Q(x) +x \frac{\partial R}{\partial C}+\frac{\partial
S}{\partial C}. $$

Then, we just need to divide $-Q$ by the same quadratic $(x^2+Bx+C)$
to get two of the
partial derivatives of $R$ and $S$:

$$\textcolor{blue}{-Q(x)=(x^2+Bx+C)\ \frac{\partial Q}{\partial C}}+\textcolor{red}{x \frac{\partial R}{\partial C}+\frac{\partial
S}{\partial C}} .$$

For $B$, we get the last two  partial derivatives:
$$\textcolor{blue}{-xQ(x)=(x^2+Bx+C)\ \frac{\partial Q}{\partial B}}+
\textcolor{red}{x \frac{\partial R}{\partial B}+\frac{\partial
S}{\partial B} }.$$

That's all.


  
\end{document}
 
