\documentclass[11pt]{amsart}
\usepackage[utf8]{inputenc}
\usepackage[T1]{fontenc}
\usepackage{lmodern}
\usepackage{amsmath}
\usepackage{amsfonts}
\usepackage{amssymb,pgfpages}
\usepackage{graphicx,boxedminipage,color,subcaption,listings,url}
\usepackage{beramono}
\usepackage{tikz}
\usepackage{tcolorbox}
\usetikzlibrary{arrows,shapes,positioning,shadows,trees}
\usepackage[french,onelanguage]{algorithm2e}
\usepackage{boxedminipage}
%\usepackage{fontspec}
%\setmainfont[Mapping=tex-text]{garamond}
%\setsansfont[Mapping=tex-text]{garamond}
%\setmonofont[Scale=0.75]{garamond}

%%\usepackage[absolute,overlay]{textpos}
%% \setlength{\TPHorizModule}{1mm}
%% \setlength{\TPVertModule}{1mm}
\usepackage{tikz}

%\usepackage[french]{babel}
%\setbeameroption{show notes}
%\setbeameroption{show notes on second screen=bottom}
\definecolor{ForestGreen}{rgb}{0.0, 0.27, 0.13}
%\setbeamercolor{frametitle}{fg = red}
\lstdefinelanguage{Algogol}%
  {morekeywords={%
      return,switch,true,type,typealias,%
      using,while},%
   sensitive=true,%
   alsoother={$},%
   morecomment=[l]\#,%
   morecomment=[n]{\#=}{=\#},%
   morestring=[s]{"}{"},%
   morestring=[m]{'}{'},%
}[keywords,comments,strings]%
\lstset{%
    language         = Algogol,
    basicstyle       = \ttfamily,
    keywordstyle     = \color{blue},
    stringstyle      = \color{magenta},
    commentstyle     = \color{ForestGreen},
    showstringspaces = false,
    keywordstyle={[2]\color{red}},
    keywordstyle={[3]\color{green}},
    keywordstyle={[4]\color{brown}},
    keywordstyle={[5]\color{gray}},
}
\lstset{
  morekeywords={[2]procedure},
  morekeywords={integer,real,array,boolean},
  morekeywords={[3]begin,end},
  morekeywords={[4]value},
  morekeywords={[5]proc,toto,bairstow,bairstowquad,bairstowstep},
}


\begin{document}

  \section*{Bairstow's Method (Sir Leonard Bairstow,
    1880--1963)}

  \begin{center}
    \textcolor{blue}{\large How to find roots of a polynomial  $P(x)$
      with real coefficients 
        \emph{without using complex arithmetic.}}
  \end{center}
   \subsection*{The idea:}    

  search for a quadratic divisor of $P$. If the degree of the quotient $Q$
  is greater than 2, reiterate with $Q$.

  
 

  Given real numbers \textcolor{blue}{$B$} and \textcolor{blue}{$C$},
  we have:

  $$P(x)= (x^2+\textcolor{blue}{B}x+\textcolor{blue}{C})\ Q(x)+ (\textcolor{blue}{R}x+\textcolor{blue}{S}),$$

  This defines a mapping:
  
$$\mathcal{F}: \begin{pmatrix}B \\ C  \end{pmatrix}
\mapsto  \begin{pmatrix}R\\S\end{pmatrix}.$$

  Cancel $\mathcal{F}(B,C)$ using Newton's method.

  \subsection*{The beauty of this method:}
  
How to compute partial derivatives of  $\mathcal{F}$? This is nice:

Once $P$ as been divided by
$(x^2+\textcolor{blue}{B}x+\textcolor{blue}{C})$, that is:
$$P(x)= (x^2+\textcolor{blue}{B}x+\textcolor{blue}{C})\ Q(x)+ (\textcolor{blue}{R}x+\textcolor{blue}{S}),$$

just cancel the first variations of $P$. This gives for $C$:

$$0 = \frac{\partial P}{\partial C}= (x^2+Bx+C)\ \frac{\partial
Q}{\partial C}+ Q(x) +x \frac{\partial R}{\partial C}+\frac{\partial
S}{\partial C}. $$

Then, we just need to divide $-Q$ by the same quadratic, and get the
partial derivatives:

$$\textcolor{blue}{-Q(x)=(x^2+Bx+C)\ \frac{\partial Q}{\partial C}}+\textcolor{red}{x \frac{\partial R}{\partial C}+\frac{\partial
S}{\partial C}} .$$

For $B$, we get:
$$\textcolor{blue}{-xQ(x)=(x^2+Bx+C)\ \frac{\partial Q}{\partial B}}+
\textcolor{red}{x \frac{\partial R}{\partial B}+\frac{\partial
S}{\partial B} }.$$

That's all.


  
\end{document}
 
